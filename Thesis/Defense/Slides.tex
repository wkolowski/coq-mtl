\documentclass{beamer}

\usepackage[utf8]{inputenc}
\usepackage{polski}
\usepackage[polish]{babel}
\usepackage{graphicx}

\usetheme{Darmstadt}
\useoutertheme{miniframes}
\makeatletter
  \beamer@compressfalse
\makeatother

\title{Formalnie zweryfikowane programowanie z monadami w Coqu}
\author{Wojciech Kołowski}
\date{10 września 2019}

\begin{document}

\frame{\titlepage}

\addtocontents{toc}{\setcounter{tocdepth}{1}}
\frame{\tableofcontents}

\section{Motywacja}

\begin{frame}{Motywacje praktyczne}
\begin{itemize}
	\item Monady bywają przydatne - parę razy potrzebowałem ich w kontekście opcji czy list, żeby za dużo nie pisać.
	\item W bibliotece standardowej Coqa nie ma monad.
	\item Próba wyszukania słowa ``monad'' na liście Coqowych pakietów \url{https://coq.inria.fr/opam/www/} nie daje żadnej sensownej odpowiedzi (choć jest kilka wystąpień).
	\item W Coqu nie ma żadnego pomysłu na zarządzanie efektami.
	\item Być może dlatego, że w Coqu nie ma żadnych ``prawdziwych'' efektów w stylu IO (choć ostatnio pojawiły się biblioteki/pakiety, które dają Coqowi IO).
\end{itemize}
\end{frame}

\begin{frame}{Motywacje dydaktyczne}
\begin{itemize}
	\item Człowiek ucząc się Haskella poznaje głównie sposób użycia monad w praktyce.
	\item Gorzej z kwestiami takimi jak prawa: choć Haskell reklamuje się jako przyjazny rozumowaniu równaniowemu, to dowodzenie praw zazwyczaj odbywa się w komentarzach albo postach na blogach.
	\item Kto normalny dowodzi praw, jeżeli typechecker go nie zmusza? Ja nie.
	\item W Coqu jest wprost przeciwnie - typechecker zmusza do dowodzenia praw, a CoqIDE umożliwia robienie tego w wygodny, interaktywny sposób.
	\item (Re)implementacja monad w Coqu daje więc dużo większe pole do nauki niż Haskell.
\end{itemize}
\end{frame}

\begin{frame}{Motywacje teoretyczne}
\begin{itemize}
	\item Prawa monad dużo lepiej wyglądają, gdy prezentuje się je za pomocą \texttt{>=>} zamiast \texttt{>>=}, więc dlaczego ludzie tak nie robią?
	\item Czy stare Haskellowe \texttt{Monad} to to samo co nowe \texttt{Monad}, którego nadklasą jest \texttt{Applicative}?
	\item Czy faktycznie wszystkie dowody przechodzą tak gładko, jak chcieliby Haskellowi miłośnicy rozumowań równaniowych?
	\item Czy na temat efektów da się w ogóle formalnie rozumować wewnątrz języka?
\end{itemize}
\end{frame}

\begin{frame}{Co i jak z tego wyszło}
\begin{itemize}
	\item Początki projektu sięgają kwietnia 2017 -- była to wówczas próba załatania braku w standardowej bibliotece Coqa różnych wygodnych rzeczy znanych z Haskella.
	\item Była to pierwsza tego typu Coqowa biblioteka albo nie umiem szukać (choć w międzyczasie powstała też podobna biblioteka z nieco innymi celami oraz używająca innych mechanizmów abstrakcji).
	\item Siła rzeczy (oraz kontakty z promotorem i seminarium z efektów algebraicznych) popychały ewolucję projektu w kierunku czegoś, co zaczęło przypominać mądrzejszą kopię Haskellowej biblioteki \texttt{mtl}.
	\item Myślę, że efekty sa całkiem dobre (przynajmniej dopóki patrzymy tylko na proste przykłady -- co by było w dużej skali, nie wiadomo).
\end{itemize}
\end{frame}

\section{Streszczenie pracy}

\begin{frame}{Do kogo adresowana jest praca?}
\begin{itemize}
	\item Wychodzę z założenia, że piszę po to, aby być czytanym.
	\item Oficjalnym celem pracy jest zgarnięcie papierka zwanego dyplomem inżyniera, a nieoficjalnym -- nauczenie czytelnika (i samego siebie) czegoś wartościowego.
	\item Ciekawym skutkiem tego podejścia jest to, że każdy rozdział (a czasem nawet pojedyncze sekcje) są adresowane do zupełnie różnych grup odbiorców, choć fakt ten może być niewidzialny.
	\item Recenzje pracy są pozytywne, więc chyba nie jest taka zła.
\end{itemize}
\end{frame}

\begin{frame}{Weryfikacja hard- i software'u}
\begin{itemize}
	\item Rozdział 1 pisałem z myślą o recenzencie zanim jeszcze dowiedziałem się, kto nim jest.
	\item Jego cel jest prosty -- uzasadnienie potrzebności i ważności interesujących mnie rzeczy za pomocą sprytnej retoryki.
	\item Jeżeli spec od formalnej weryfikacji chce uzasadnić swoje istnienie, musi wspomnieć coś o spadających samolotach.
	\item Koniunktura sprzyja takiemu postawieniu sprawy, gdyż ostatnimi czasy namnożyło się bardzo poważnych błędów, jak Meltdown, Spectre czy Heartbleed. Spadające samoloty też były, więc retoryka jest nawet zgodna z prawdą.
\end{itemize}
\end{frame}

\begin{frame}{Weryfikacja matematyki}
\begin{itemize}
	\item Podrozdział 1.2 to próba przemówienia do rozumu matematykom.
	\item Matematykom wydaje się, że logika jest sztuczna i nie odpowiada praktyce, a formalizacja nie jest warta zachodu.
	\item Klasycznym kontrprzykładem jest dowód twierdzenia o czterech barwach, wymagający sprawdzenia setek przypadków.
	\item Innym ciekawym przypadkiem są perypetia Vladimira Voevodskyego, laureata medalu Fieldsa, który w 1989 opublikował pewną pracę o $\infty$-grupoidach i teorii homotopii. Znalezienie w niej błędu zajęło innemu ekspertowi 9 lat, ale przez kolejne 15 lat Voevodsky nie wierzył w jego argumenty.
\end{itemize}
\end{frame}

\begin{frame}{Efekty}
\begin{itemize}
	\item Rozdział 2 pisałem z myślą o typowym programiście, który niespecjalnie zdaje sobie sprawę z konieczności mądrego zarządzania efektami.
	\item W sekcjach 2.1 i 2.2 staram się wyjaśnić podstawowe pojęcia i prezentuję dość prymitywny przykład.
	\item Celem tego jest zbudowanie jakiejś prostej ideologii, przez pryzmat której zwykły programista może postrzegać to, co dzieje się w jego języku.
	\item W sekcji 2.3 porównuję 3 podejścia do efektów:
	\begin{itemize}
		\item Monady, transformatory, klasy efektów (jak np. \texttt{MonadState} etc.) i reszta znana z \texttt{mtl}a.
		\item \texttt{extensible-effects}
		\item Efektry algebraiczne, czyli najnowszy krzyk mody.
	\end{itemize}
\end{itemize}
\end{frame}

\begin{frame}{Abstrakcja i inżynieria dowodu}
\begin{itemize}
	\item Większa część rozdziału 3 (3.1, 3.2, 3.3) jest adresowana do mnie samego. Strasznie irytuje mnie mnogość powiązanych mechanizmów abstrakcji dostępna w Coqu i to, że żadne prace (o ile mnie pamięć nie myli) nie tłumaczą swojego wyboru mechanizmów abstrakcji. Musiałem coś z tym zrobić -- wyszło porównanie klas typów, struktur kanonicznych i modułów.
	\item Pozostała część rozdziału (3.4 i 3.5) była adresowana do inżynierów oprogramowania. Jej celem było podkreślenie roli nowej dziedziny inżynierii, która wyłania się z formalizacji -- inżynierii oprogramowania. Tu aksjomaty stają się częścią architektury, a automatyzacja dowodów -- czymś analogicznym np. do systemów ORM (choć porównanie jest być może naciągane). 
\end{itemize}
\end{frame}

\begin{frame}{Implementacja}
\begin{itemize}
	\item Rozdział 4 skierowany jest głównie do wnikliwego czytelnika, który zna się na rzeczy, np. recenzenta.
	\item Jego celem jest sztampowy opis implementacji połączony z innowacyjnym opisem różnych dziwnych rzeczy, które odkryłem, a których się nie spodziewałem.
	\item Okazuje się, że znane z Haskella klasy wymagają nowych prawy (bo naturalność nie działa), a te sprawiają np., że nowe zestawy praw nie są minimalne -- niektóre prawa da się wyprowadzić z innych.
\end{itemize}
\end{frame}

\begin{frame}{Inne odkrycia}
\begin{itemize}
	\item Trzeba było też zrewidować kilka metod implementacji. Dla przykładu, wolne monady znane z Haskella nie są legalne w Coqu, więc trzeba posiłkować się kodowaniem Churcha.
	\item Ciekawym doświadczeniem było empiryczne poszukiwanie praw dla klasy \texttt{WriterT}.
	\item Były też i porażki -- np. nie udało mi się pokazać, że \texttt{FreeT} spełnia prawa \texttt{MonadExcept}, czyli efektu wyjątków.
\end{itemize}
\end{frame}

\end{document}
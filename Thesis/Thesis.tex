% Opcje klasy 'iithesis' opisane sa w komentarzach w pliku klasy. Za ich pomoca
% ustawia sie przede wszystkim jezyk i rodzaj (lic/inz/mgr) pracy, oraz czy na
% drugiej stronie pracy ma byc skladany wzor oswiadczenia o autorskim wykonaniu.
\documentclass[declaration,inz,english,shortabstract]{iithesis}

\usepackage[utf8]{inputenc}

%%%%% DANE DO STRONY TYTULOWEJ
% Niezaleznie od jezyka pracy wybranego w opcjach klasy, tytul i streszczenie
% pracy nalezy podac zarowno w jezyku polskim, jak i angielskim.
% Pamietaj o madrym (zgodnym z logicznym rozbiorem zdania oraz estetyka) recznym
% zlamaniu wierszy w temacie pracy, zwlaszcza tego w jezyku pracy. Uzyj do tego
% polecenia \fmlinebreak.
\englishtitle   {Formally verified programming with monads in Coq}
\polishtitle    {Formalnie zweryfikowane programowanie z monadami w Coqu}
\polishabstract {\ldots}
\englishabstract{We introduce \libname, a Coq library for formally verified general-purpose programming with Haskell-style abstractions: functors, applicatives, monads, monad transformers and class-based monads. We discuss the design choices made and illustrate the working of the library with examples from \cite{JustDoIt}.}
% w pracach wielu autorow nazwiska mozna oddzielic poleceniem \and
\author         {Zeimer}
% w przypadku kilku promotorow, lub koniecznosci podania ich afiliacji, linie
% w ponizszym poleceniu mozna zlamac poleceniem \fmlinebreak
\advisor        {dr Wpisuyashi TODO}
\date           {czerwiec 2019}                     % Data zlozenia pracy
% Dane do oswiadczenia o autorskim wykonaniu
%\transcriptnum {}                     % Numer indeksu
%\advisorgen    {dr. Jana Kowalskiego} % Nazwisko promotora w dopelniaczu
%%%%%

%%%%% WLASNE DODATKOWE PAKIETY
%
%\usepackage{graphicx,listings,amsmath,amssymb,amsthm,amsfonts,tikz}
%
%%%%% WLASNE DEFINICJE I POLECENIA
%
%\theoremstyle{definition} \newtheorem{definition}{Definition}[chapter]
%\theoremstyle{remark} \newtheorem{remark}[definition]{Observation}
%\theoremstyle{plain} \newtheorem{theorem}[definition]{Theorem}
%\theoremstyle{plain} \newtheorem{lemma}[definition]{Lemma}
%\renewcommand \qedsymbol {\ensuremath{\square}}

\newcommand{\libname}{hsCoq}

%%%%%

\begin{document}

%%%%% POCZATEK ZASADNICZEGO TEKSTU PRACY

\chapter{Introduction}
 
This is the first section.sss

\chapter{About Coq}

wuuut

\chapter{Computational effects}

\chapter{Design}

\chapter{Examples}

\chapter{A case study in proof engineering}

\chapter{Conclusion}

\chapter{TODO}

\begin{enumerate}
    \item Introduction: functional programming, formally verified programming and proving.
    \item Approaches to computational effects: chaos, ML-style, monads, algebraic effects.
    \item A description of the inner workings of the library: design choices, file structure, implementation.
    \item Examples: some from Just Do It, maybe some custom ones.
    \item Safety: some theorems and proofs.
    \item Theoretical comparison of the ease of use with Haskell and Idris.
    \item Practical comparison with MERC.
    \item Cite some literature: some Coq papers, Moggi, Just Do It, Experimenting with Monadic Equational Reasoning in Coq
    \item Technical matters:
    \begin{enumerate}
        \item Mention where's the implementation and put it to Coq's repository of user libraries.
        \item Installation guide.
        \item Tools: why no ssreflect?
        \item Documentation (it's in the source code).
    \end{enumerate}
    \item More: a case study in proof engineering - how do the tactics hs, monad and (maybe) the one for reflective functor simplifcation work?
    \item Deficiencies, conclusion and further work.
    \item Points to make: this is a library for general purpose programming, without some deep goal.
\end{enumerate}

%%%%% BIBLIOGRAFIA

\begin{thebibliography}{1}

    \bibitem{JustDoIt}
      Jeremy Gibbons and Ralf Hinze,
      \textit{Just do It: Simple Monadic Equational Reasoning},
      2011
    
\end{thebibliography}

%\begin{thebibliography}{1}
%\bibitem{example} \ldots
%\end{thebibliography}

\end{document}